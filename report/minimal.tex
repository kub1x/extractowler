\input ctustyle
\worktype [M/CZ]
\faculty {F3}
\department {Katedra kybernetiky}
\title {Minimální dokument}
\author {Jakub Podlaha}
\date {November 2013}
\abstractEN {This document is for testing purpose only.}
\abstractCZ {Tento dokument je pouze pro potřeby testování.}
\declaration {Prohlašuji, že jsem se neflákal.}
\makefront

\chap Zadání SW Projektu

\begitems \style n
  * Seznamte se technologiemi pro automatickou extrakci dat z webových stránek
    a s jazyky sémantického webu RDF, RDFS a OWL.
  * Navrhněte a implementujte vhodný datový formát pro popis scénářů extrakce
    dat, které bude možné zpracovat vhodným open-source crawlerem (např.  [1]).
    Vytvořte jednoduché uživatelské rozhraní ve vhodném webovém prohlížeči,
    sloužící k tvorbě scénářů ve vámi navrženém datovém formátu pro následnou
    extrakci sémantických data z webových stránek. 
\enditems


\chap Knowledge base, principles and technologies

Seznamte se technologiemi pro automatickou extrakci dat z webových stránek a s
jazyky sémantického webu RDF, RDFS a OWL.

%\sec automatická extrakce dat

\sec RDF and RDFS

\begitems
 * \url{https://en.wikipedia.org/wiki/Resource_Description_Framework}
\enditems


\sec OWL

\begitems
 * \url{http://www.w3.org/TR/owl2-primer/}
 * \url{https://en.wikipedia.org/wiki/Web_Ontology_Language}
 * \url{http://www.w3.org/TR/2012/REC-owl2-quick-reference-20121211/}
\enditems


\sec Linked Data

\begitems
 * \url{http://linkeddata.org/guides-and-tutorials}
 * \url{http://linkeddatabook.com/editions/1.0/}
 * \url{http://lov.okfn.org/dataset/lov/}
\enditems


\sec Ontology repositories

\begitems
 * \url{http://www.w3.org/wiki/Ontology_repositories}
\enditems


\sec RDFa

\begitems
 * \url{https://www.sio2.cz/web/psiotwo/publications}
 * \url{http://rdfa.info/play/}
\enditems


\sec dalsi

\begitems
 * \url{https://en.wikipedia.org/wiki/SPARQL}
 * \url{https://en.wikipedia.org/wiki/Turtle_(syntax)}
\enditems


\chap research - existující řešení


\sec InfoCram 2000 - Jirka

\begitems
  * zalozeny na Aardwark \urlnote{https://addons.mozilla.org/en-US/firefox/addon/aardvark/}
\enditems


\sec iMacros

\begitems
  * \url{http://wiki.imacros.net/Command_Reference}
  * \url{http://wiki.imacros.net/iMacros_for_Firefox}
  * \url{http://wiki.imacros.net/iMacros_for_Chrome}
\enditems


\sec Sahi

Yet another web automation project. \url{http://sourceforge.net/projects/sahi/}


\sec Selenium IDE

\begitems
  * IDE - \url{http://www.seleniumhq.org/projects/ide/}
  * plugins - \url{http://www.seleniumhq.org/projects/ide/plugins.jsp}
  * current commands - \url{http://release.seleniumhq.org/selenium-core/1.0.1/reference.html}
  * documentation - \url{http://docs.seleniumhq.org/docs/index.jsp}
  * extending selenium API (blog, tutorial) - \url{http://adam.goucher.ca/?s=selenium&paged=2}
  \begitems
    * randomString example - \url{http://adam.goucher.ca/?p=1348}
  \enditems
\enditems






\chap crOWLer

\sec zavislosti

\begitems \style O
  * maven - apache project managing tool
  \begitems \style o
    * \url{https://maven.apache.org}
    * \url{https://maven.apache.org/run-maven/index.html}
    * \url{https://maven.apache.org/guides/mini/guide-ide-eclipse.html}
  \enditems
  * sesame
  \begitems \style o
    * \url{http://www.openrdf.org/download.jsp} ??
  \enditems
  * jena
  \begitems \style o
    * \url{https://github.com/ansell/JenaSesame} !!
    * or \url{https://github.com/afs/JenaSesame} ??
    * or \url{http://jena.apache.org/} ???
    * or \url{http://sjadapter.sourceforge.net/} ????
    * or \url{http://sourceforge.net/projects/jenasesamemodel/}
    * might help \url{http://www.iandickinson.me.uk/articles/jena-eclipse-helloworld/}
    * little hint \url{http://spqr.cerch.kcl.ac.uk/?page_id=130}
    * another hit \url{http://answers.semanticweb.com/questions/20865/how-to-get-the-jena-sesame-adapter}
    * wiki \url{https://en.wikipedia.org/wiki/Jena_(framework)}
    * jena vs. sesame flame \url{http://answers.semanticweb.com/questions/1638/jena-vs-sesame-is-there-a-serious-complete-up-to-date-unbiased-well-informed-side-by-side-comparison-between-the-two}
  \enditems
\enditems


\sec Implementation


\secc Classes of CrOWLer

\begitems
  * ImmovableHeritageConfiguration extends MonumnetConfiguration implements ConfigurationFactory 
  \begitems
    * implements Configuration, which is parameter for FullCrawler.run() method
  \enditems
  * FullCrawler
  \begitems
    * implements the whole crawling algorithm
    * 
  \enditems
\enditems


\sec notes

\begitems
  * \url{http://onto.mondis.cz/resource/page/npu/}
\enditems


\secc Run configuration 

\begtt
crowler cz.sio2.crowler.configurations.npu.ImmovableHeritageConfiguration file results
crowler cz.sio2.crowler.configurations.kub1x.KbxConfiguration file results
crowler cz.sio2.crowler.configurations.parser.SeleniumConfiguration\
         file results generated.html
\endtt

\begitems
  * Class ImmovableHeritageConfiguration implements Configuration class. 
  * Folder jena\_con will be created and all the rdf's will be stored in int with names derived from ontology uri
\enditems



\chap Data


\sec Pamatky

\begitems
  * \url{http://monumnet.npu.cz/pamfond/list.php?hledani=1&KrOk=&HiZe=&VybUzemi=1&sNazSidOb=&Adresa=&Cdom=&Pamatka=&CiRejst=&Uz=B&PrirUbytOd=3.5.1958&PrirUbytDo=10.12.2013}
  * \url{http://dominanty.cz/pamatky-cihana.php}
\enditems



\chap Implementace


\sec Ideas


\secc Overlay on webpage

\begitems
  * create an overlay that will highlight information being crowled
  * the rest of webpage will gray out
  * will show classes of each highlighted region
  * onmouseover will show arrows with relations aswell
  * will show current context in a table view aswell
\enditems


\secc Selenium Builder - new technology

\url{https://github.com/sebuilder/se-builder/wiki/Getting-Started}


\sec SelectOWL - Plugin pro Selenium IDE - Firefox

\begitems
  * \url{https://developer.mozilla.org/en-US/Add-ons/Setting_up_extension_development_environment}
  * \url{http://kb.mozillazine.org/Getting_started_with_extension_development}
  * \url{http://code.google.com/p/selenium/source/browse/}
  * \url{http://docs.seleniumhq.org/download/maven.jsp}
  * \url{http://repo1.maven.org/maven2/org/seleniumhq/selenium/ide/selenium-ide/1.0.2/}
\enditems



\sec Snippets

\begitems
  * \url{https://code.google.com/p/selenium/source/browse/ide/main/src/content/testCase.js} - definition of TestCase and Command (!!!)
  * selenium/chrome/content/selenium/scripts/selenium-commandhandlers.js - registrace prikazu (vytvari se tam "AndWait" postfixy etc.)
\enditems


\secc Embeding selenium-commandhandlers.js

I need to embed the selenium core itself in order to add new TYPE of commands. 

The original selenium commands are: 

\begitems
  * accessors (i.e. getSometing or isSomething -> getFoo, assertFoo, verifyFoo, assertNotFoo, verifyNotFoo, storeFoo, waitForFoo, and waitForNotFoo.
  * asserts (i.e. assertSomething -> assertSomething)
  * actions (i.e. doSomeAction -> someAction, someActionAndWait)
\enditems

none of which applies for owl commands that are more

\begtt
vim selenium/chrome/content/selenium/scripts/selenium-commandhandlers.js
\endtt

\begtt
vim selenium/chrome/content/selenium/scripts/htmlutils.js
21: function classCreate()
  - constructor that calls initialize on self with arguments passed
27: function objectExtend(destination, source)
\endtt

\begtt
vim selenium/chrome/content/selenium/scripts/selenium-executionloop.js
 94: _executeCurrentCommand - calls:
104: var handler = this.commandFactory.getCommandHandler(command.command);
112: this.result = handler.execute(selenium, command);
\endtt

HANDLER HAS TO IMPLEMENT execute(seleniumApi, commandObj);



\chap Bookmarklet prototype


\sec step by step

\begitems
  * create bookmarklet to alert from external script
  * fill it with simple \<div\> containing an external html
  * insert form to load a file or url with ontology
  * add jOWL to load the ontology
  * add jOWLBrowser-ish functionality to visualize the ontology
  * add aardwark.js to select items
  * create the json expressing scripts for crowler
  * visualise anotated data
  * export and run in crowler
\enditems


\bye
