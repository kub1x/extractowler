\bibchap
%\ifx\usebib\undefined
%   \usebbl/c mybase
%\else
\usebib/c (simple) mybase
%\fi



\label[abbreviations]
\app Abbreviations %and symbols

\abbrv[MDN] Mozilla Developers Network
\abbrv[URI] Uniform Resource Identifier
\abbrv[URL] Uniform Resource Locator
\abbrv[URN] Uniform Resource Name
\abbrv[RDF] Resource Description Framework
\abbrv[RDFS] RDF Schema
  - set of classes and properties providing basic elements for the description of ontologies
\abbrv[OWL] Web Ontology Language
\abbrv[SPARQL] SPARQL Protocol and RDF Query Language - query language for semantic databases/triplestores
\abbrv[foaf] friend of a friend - a popular ontology for describing personal information and relationships



%\font\mflogo=logo10 at11pt
%\def\METAFONT{{\mflogo METAFONT}}
%\def\METAPOST{{\mflogo METAPOST}}
%
%Tento text je až na výjimky převzat z~\cite[zyka].
%
%\sec Zkratky
%
%Jako příklad pro popis zkratek poslouží pojmy ze světa \TeX{}u.
%
%\medskip
%\bgroup \leftskip=6.3em
%\abbrv[\TeX{}]  Program na přípravu elektronické sazby vysoké kvality
%   vytvořený Donaldem Knuthem. Program zahrnuje interpret makrojazyka.
%   Název programu se vyslovuje \uv{tech}.
%\abbrv[\METAFONT{}] Program a makro jazyk pro generování fontů
%   z vektorového do bitmapvého formátu vytvořený Donaldem Knuthem.
%\abbrv[\METAPOST{}] Program generující vektorovou grafiku založený na
%   \METAFONT{}u vytvořený Johnem Hobby.
%\abbrv[plain\TeX{}]  Originální \TeX{}ový formát (rošiření na úrovni 
%   makrojazyka). Je součástí každé distribuce \TeX{}u a je
%   vytvořen Donaldem Knuthem.
%\abbrv[\csplain{}] \TeX{}ový formát rozšiřující plain\TeX{} o možnosti sazby
%   v českém a slovenském jazyce vytvořený Petrem Olšákem.
%\abbrv[\LaTeX{}]  Nejznámější \TeX{}ový formát (rozšíření na úrovni 
%   makrojazyka) vytvořený Leslie Lamportem. 
%   Existuje obludné množství různých balíčků, které pomocí
%   makrojazyka \TeX{}u dále rozšiřují výchozí možnosti \LaTeX{}u.
%   Rozličné uživatelské požadavky jsou nejčastěji řešeny použitím vhodného balíčku.
%\abbrv[OPmac] Olšákovy Plain\TeX{}ová makra nabízející uživatelům
%   plain\TeX{}u podobné možnosti, jako \LaTeX{}, ovšem přímočařeji
%   a jednodušeji.
%\abbrv[Con\TeX{}t]  Typografický systém vystavěný na rozšíření \TeX{}u s
%   názvem Lua\TeX{} (kombinuje makrojazyk \TeX{}u s jazykem Lua) a na množství
%   předpřipravených makro souborů vytvořený týmem v čele s Hansem Hagenem.
%   Rozličné uživatelské požadavky jsou nastavovány pomocí přiřazení hodnot
%   klíčovým slovům společně s možností \TeX{}ového, \METAPOST{}ího a 
%   Lua programování.
%\par\egroup
%
%
%\sec Symboly
%
%\medskip
%\bgroup \leftskip=2em
%\abbrv[$\pi$] Konečná verze \TeX{}u zmíněna v Knuthově \TeX{}tamentu.
%\abbrv[e] Konečná verze \METAFONT{}u.

