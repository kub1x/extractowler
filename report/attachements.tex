\bibchap
%\ifx\usebib\undefined
%   \usebbl/c mybase
%\else
\usebib/c (simple) mybase
%\fi



\app Assignment 

\picw=15cm\cinspic zadani.png






\label[abbreviations]
\app Abbreviations %and symbols

\abbrv[MDN] Mozilla Developers Network
\abbrv[URI] Uniform Resource Identifier
\abbrv[URL] Uniform Resource Locator
\abbrv[URN] Uniform Resource Name
\abbrv[RDF] Resource Description Framework
\abbrv[RDFS] RDF Schema
  - set of classes and properties providing basic elements for the description of ontologies
\abbrv[OWL] Web Ontology Language
\abbrv[SPARQL] SPARQL Protocol and RDF Query Language - query language for semantic databases/triplestores
\abbrv[foaf] friend of a friend - a popular ontology for describing personal information and relationships






\label[app:rdfsvocab]
\app RDF and RDFS vocabulary

\midinsert \clabel[rdfsvocab]{RDF and RDFS vocabulary}
\ctable{rl}{
\hfil resource   & description  \crl \tskip4pt
  rdf:type       & a property used to state that a resource is an instance of a class \cr
                 & a commonly accepted qname for this property is {\tt a} \cr
  rdfs:Resource  & the class of everything; all things described by RDF are resources \cr
  rdfs:Class     & declares a resource as a class for other resources \cr
  rdfs:Literal   & literal values such as strings and integers \cr
                 & property values such as textual strings are examples of RDF literals\cr
                 & literals may be plain or typed \cr
  rdfs:Datatype  & the class of datatypes\cr
                 & rdfs:Datatype is both an instance of and a subclass of rdfs:Clas \cr
                 & each instance of rdfs:Datatype is a subclass of rdfs:Literal \cr
  rdf:XMLLiteral & the class of XML literal values; rdf:XMLLiteral is an instance \cr
                 & of rdfs:Datatype (and thus a subclass of rdfs:Literal) \cr
  rdf:Property   & the class of properties \cr
  rdfs:domain    & (of an rdf:predicate) declares the class of the subject in a triple \cr
                 & whose second component is the predicate \cr
  rdfs:range     & (of an rdf:predicate) declares the class or datatype of the object in a triple \cr
                 & whose second component is the predicate \cr
  rdfs:subClassOf    & allows to declare hierarchies of classes \cr
  rdfs:subPropertyOf & an instance of rdf:Property that is used to state \cr
                     & that all resources related by one property are also related by another \cr
  rdfs:label     & rdf:Property used to provide a human-readable version of a resource's name \cr
  rdfs:comment   & rdf:Property used to provide a human-readable description of a resource \cr
}
\caption/t RDF and RDFS vocabulary
\endinsert





\label[app:rdfxml]
\app Example of RDF/XML syntax

% Taken directly from the FOAF ontology

\begtt
<rdf:RDF xmlns:rdf="http://www.w3.org/1999/02/22-rdf-syntax-ns#" 
         xmlns:rdfs="http://www.w3.org/2000/01/rdf-schema#" 
         xmlns:owl="http://www.w3.org/2002/07/owl#" 
         xmlns:vs="http://www.w3.org/2003/06/sw-vocab-status/ns#" 
         xmlns:foaf="http://xmlns.com/foaf/0.1/" 
         xmlns:dc="http://purl.org/dc/elements/1.1/">
  <!-- Here we describe general characteristics
       of the FOAF vocabulary ('ontology'). -->
  <owl:Ontology rdf:about="http://xmlns.com/foaf/0.1/"
                dc:title="Friend of a Friend (FOAF) vocabulary"
                dc:description="The Friend of a Friend (FOAF) RDF
                                vocabulary, described using
                                W3C RDF Schema and OWL the Web
                                Ontology Language." >
  </owl:Ontology>
  <rdfs:Class rdf:about="http://xmlns.com/foaf/0.1/Person"
              rdfs:label="Person"
              rdfs:comment="A person."
              vs:term_status="stable">
    <rdf:type rdf:resource="http://www.w3.org/2002/07/owl#Class"/>
    <owl:equivalentClass
      rdf:resource="http://schema.org/Person" />
    <owl:equivalentClass
      rdf:resource="http://www.w3.org/2000/10/swap/pim/contact#Person"/>
    <rdfs:subClassOf>
      <owl:Class rdf:about="http://xmlns.com/foaf/0.1/Agent"/>
    </rdfs:subClassOf>
    <rdfs:subClassOf>
      <owl:Class
        rdf:about="http://www.w3.org/2003/01/geo/wgs84_pos#SpatialThing"
        rdfs:label="Spatial Thing"/>
    </rdfs:subClassOf>
    <rdfs:isDefinedBy
      rdf:resource="http://xmlns.com/foaf/0.1/"/>
    <owl:disjointWith
      rdf:resource="http://xmlns.com/foaf/0.1/Organization"/>
    <owl:disjointWith
      rdf:resource="http://xmlns.com/foaf/0.1/Project"/>
  </rdfs:Class>
  <!-- (...) -->
</rdf:RDF>
\endtt





\label[app:crowler-new-component]
\app crOWLer architecture

~ 

\picw=12cm\cinspic crowler-new-component.png





\label[app:sowljson]
\app SOWL/JSON scenario solving Use Case 1

\begtt
{
  type: "scenario", 
  name: "manual", 
  ontology: {
    base: "http://kub1x.org/onto/dip/t/", 
    imports : [
      {
        prefix: "foaf", 
        uri: "http://xmlns.com/foaf/0.1/", 
      }, 
      {
        prefix: "kbx", 
        uri: "http://kub1x.org/onto/dip/t/", 
      }, 
    ], 
  }, 
  creation-date: "2014-11-30 12:40", 
  call-template: {
    command: "call-template", 
    name: "init", 
    url: "http://www.inventati.org/kub1x/t/", 
  }, 
  templates: [
    {
      name: "init", 
      steps: [
        {
          command: "onto-elem", 
          typeof: "http://xmlns.com/foaf/0.1/Person", 
          selector: {
            value: "tr", 
            type: "css", 
          }, 
          steps: [
            {
              command: "value-of", 
              property: "http://xmlns.com/foaf/0.1/firstName", 
              selector: {
                value: "td:nth-child(1)", 
                type: "css", 
              }, 
            }, 
            {
              command: "value-of", 
              property: "http://xmlns.com/foaf/0.1/lastName", 
              selector: {
                value: "td:nth-child(2)", 
                type:  "css", 
              }, 
            }, 
            {
              command: "value-of", 
              property: "http://xmlns.com/foaf/0.1/phone", 
              selector: {
                value: "td:nth-child(3)", 
                type:  "css", 
              }, 
            },
            {
              command: "call-template", 
              name: "detail", 
              selector: {
                value: [
                  {
                    value: "td.detail a", 
                    type:  "css", 
                  }, 
                  {
                    value: "@href", 
                    type: "xpath", 
                  }, 
                ], 
                type: "chained", 
              }, 
            }, 
          ], 
        }, 
      ], 
    }, 
    {
      name: "detail", 
      steps: [
        {
          command: "value-of", 
          property: "http://xmlns.com/foaf/0.1/nickname", 
          selector: {
            value: ".nick", 
            type:  "css", 
          }, 
        },
      ], 
    }, 
  ], 
}
\endtt




%\font\mflogo=logo10 at11pt
%\def\METAFONT{{\mflogo METAFONT}}
%\def\METAPOST{{\mflogo METAPOST}}
%
%Tento text je až na výjimky převzat z~\cite[zyka].
%
%\sec Zkratky
%
%Jako příklad pro popis zkratek poslouží pojmy ze světa \TeX{}u.
%
%\medskip
%\bgroup \leftskip=6.3em
%\abbrv[\TeX{}]  Program na přípravu elektronické sazby vysoké kvality
%   vytvořený Donaldem Knuthem. Program zahrnuje interpret makrojazyka.
%   Název programu se vyslovuje \uv{tech}.
%\abbrv[\METAFONT{}] Program a makro jazyk pro generování fontů
%   z vektorového do bitmapvého formátu vytvořený Donaldem Knuthem.
%\abbrv[\METAPOST{}] Program generující vektorovou grafiku založený na
%   \METAFONT{}u vytvořený Johnem Hobby.
%\abbrv[plain\TeX{}]  Originální \TeX{}ový formát (rošiření na úrovni 
%   makrojazyka). Je součástí každé distribuce \TeX{}u a je
%   vytvořen Donaldem Knuthem.
%\abbrv[\csplain{}] \TeX{}ový formát rozšiřující plain\TeX{} o možnosti sazby
%   v českém a slovenském jazyce vytvořený Petrem Olšákem.
%\abbrv[\LaTeX{}]  Nejznámější \TeX{}ový formát (rozšíření na úrovni 
%   makrojazyka) vytvořený Leslie Lamportem. 
%   Existuje obludné množství různých balíčků, které pomocí
%   makrojazyka \TeX{}u dále rozšiřují výchozí možnosti \LaTeX{}u.
%   Rozličné uživatelské požadavky jsou nejčastěji řešeny použitím vhodného balíčku.
%\abbrv[OPmac] Olšákovy Plain\TeX{}ová makra nabízející uživatelům
%   plain\TeX{}u podobné možnosti, jako \LaTeX{}, ovšem přímočařeji
%   a jednodušeji.
%\abbrv[Con\TeX{}t]  Typografický systém vystavěný na rozšíření \TeX{}u s
%   názvem Lua\TeX{} (kombinuje makrojazyk \TeX{}u s jazykem Lua) a na množství
%   předpřipravených makro souborů vytvořený týmem v čele s Hansem Hagenem.
%   Rozličné uživatelské požadavky jsou nastavovány pomocí přiřazení hodnot
%   klíčovým slovům společně s možností \TeX{}ového, \METAPOST{}ího a 
%   Lua programování.
%\par\egroup
%
%
%\sec Symboly
%
%\medskip
%\bgroup \leftskip=2em
%\abbrv[$\pi$] Konečná verze \TeX{}u zmíněna v Knuthově \TeX{}tamentu.
%\abbrv[e] Konečná verze \METAFONT{}u.

