\label[chap:implementation] 
\chap Program Implementation

This chapter describes the implemented prototype.

TODO

\midinsert \clabel[pic:sowl-crowler-stack]{Overview of the whole stack and files exchanged}
\picw=12cm \cinspic sowl-crowler-stack.png
\caption/f Overview of the whole stack and files exchanged
\endinsert

\sec Parsing Ontologies in JavaScript

TODO jOWL vs rdfQuery


TODO aardvark

TODO TODO TODO

In current Implementation of SOWL we create different type of Firofox extension
using new SDK. Moreover the aardvark code is injected directly into the webpage
using the Content Script feature of Firefox SDK. According to these differences
the bookmarklet version better fits the needs and is used. The code was included
in the addon and extended with features necessary for SOWL. 

TODO TODO TODO



\sec Scenario format

One of main tasks of this work was to create format for scenario generated by
SOWL and consumed by crOWLer. This scenario will describe information necessary
for the crawling process: what operation to do (create ontological object,
assign property to such an object, perform task with webpage). 

This task is closely related to implementation peculiarity of semantic crawler:
we're dealing with two separate contexts at the same time, the ontological and
the web context. Ontological context holds current object (individual) to which
we assign properties, web context hold current webpage along with currently
selected element on that webpage. Scenario have to support operations to change
each context separately and/or both at the same time. 



\secc Strigil/XML

Strigil, the scraping platform in order to solve similar problem as crOWLer
introduces it's own XML based Scraping Script format. It's documentation can be
found here XXX
\urlnote{https://drive.google.com/file/d/0B4On-lGb38CgWlAyZDhGbDV2TFk/edit
Scraping script documentation}. 

Basis of the whole script is system of {\em templates}. Each template has a name
and mime-type declaring type of document the template is designed for. This
information is needed as Strigil supports HTML and also Excel spreadsheet
files. Templates call each other using {\tt call-template} command anywhere in
the script. This command accepts URL as an argument from its nested commands.
Each template is called only with new URL, thus on new document. Of course URL
of current document can be passed as an argument, but due to nature of Strigil,
this would create completely separate context. 

Strigil is tailor made for parallel processing. The architecture of the Strigil
system contains not only scraping processor, but also a layer for distributed
download queue processing and layer of proxy servers that can be used to spread
the traffic and scale the download process horizontally. As the downloads are
performed asynchronously and can be even delayed due to network lags and
timeouts, there is no guaranteed order in which documents will be scraped.
Each of Strigil templates create it's own context when called. If we want to
link data obtained from different template calls we have to use some additional
techniques. For example we can assign some properly defined, non-random, unique
identifiers to an object. This identifier have to be guaranteed to be the same
for the same object through different template calls and potentially on
different pages. 

To handle ontological data manipulation the commands {\tt onto-elem} and {\tt
value-of} are used. First one creates an individual of given type and, if
nested into different {\tt onto-elem} relates this new individual to it's
parent with some property. Literals are assigned to properties of parent object
using {\tt value-of} command with property name specified. This command is very
powerful with usage regular expressions, selectors or nested calls of itself it
can create arbitrary values from constants and data obtained from web page
being processed. 

Strigil also implements variety of functoins to help with processing of textual
data. Function {\tt addLanguageInfo}, for example, is widely used in Strigil
scraping scripts to add language tags to string literals. The function call can
be seen below. 

\begtt
<scr:function name="addLanguageInfo">
  <scr:with-param>
    <scr:value-of select="Hello World" />
  </scr:with-param>
  <scr:with-param>
    <scr:value-of text="en" />
  </scr:with-param>
</scr:function>
\endtt

Similarly we can use function {\tt addDataTypeInfo} to add datatype flag,
function {\tt generateUUID} to obtain unique identifier or function {\tt
convertDate} to convert Czech and English dates into a common {\tt xsd:date}
format and several others. Some functions, like the last one mentioned, cover
task-specific issues and Strigil doesn't define a way to extend the list of
functions. 

In early stages of SOWL development an attempt was made to use original
Strigil/XML as a format of choice. An appropriate, consistent subset was chosen
that would cover required use cases. Implementation of simple use cases
revealed some pitfalls of this decision and revealed several suggestions for
improvements on the approach and the format itself. 


\secc Adaptation of Strigil/XML format

Strigil creates it's scraping script internally hidden under GUI and leaves
user unaware of it's actual content. It might still serve well, at least for
developers, to keep the script compact and easily readable. Addition of
language tag as seen in previous chapter, is widely used pattern that polutes
the resulting script with unnecessary overload. Suggested improvement would
separate this functionality into an extra attribute of the {\tt value-of} tag
named {\tt lang}.

The same suggestion can be applied to the data-type specification. Moreover
{\em implicit parsing} of known data types would not only simplify the scraping
script, but also help to clean and clear the resulting data. 

Let's imagine hypothetical scenario of two similar tables on one page
containing two sets of data in the same format. For such a case we would need
to define a template on subset of DOM and call it twice with different root
node. Creation of {\tt dom-template} and {\tt call-dom-template} tags would
solve this issue and would allow scenario creator to narrow down his focus to a
subpart of the scraped webpage. This would be particularly useful on
complicated pages with a lot of nested HTML. {\tt dom-template} and {\tt
call-dom-template} would be defined within a single {\tt template} tag and
unlike {\tt call-template}, they would {\em keep} the ontological context co
call of {\tt value-of} within {\tt dom-template} would assign a property to
individual created by {\tt onto-elem} wrapping the current {\tt
call-dom-template} call. 

The architecture of Strigil (distributed downloader) suggests that it uses
simple raw HTML pages as they were downloaded and uses JSOUP to extract data
from it as JSOUP is the selector system of choice. Many webpages, or even web
applications, make use of dynamic AJAX calls to fetch additional data after the
presentation layer of the web is shown to the user. Strigil doesn't handle
these cases by default. The internal AJAX code could be analyzed and simulated
using {\tt call-template} call, but this requires deep knowledge of the webpage
being processed. In crOWLer we opted to switch from JSOUP to WebDriver library
and use PhantomJs, a no-GUI web browser. This technology allow us to handle
webpages the same way as user sees them. 

Usage of actual full-stack web browser with JavaScript engine long with
WebDriver allows us to inject and execute arbitrary JavaScript code into the
processed webpage. In order to make full use of this feature we can define {\tt
function-def} tag which would define JavaScript function with name and
parameters and contain it's code. To execute this function we would call {\tt
function-call} and identify it by it's name. Return value of this function can
be then used the same way as the one from {\tt value-of} tag. 

From the experience with development on Strigil/XML we can derive, that it is
tied with it's intended use for distributed downloader and it lacks some
functionality. In SOWL we would almost necessarily modify it's formal definition
and thus it is of consideration if we can't make use of more appropriate
format. 

% Strigil wiki on sf \url{http://sourceforge.net/p/strigil/home/Home}

\secc SOWL/JSON

As all Firefox extensions, SOWL is written entirely using JavaScript with
additional HTML defining the graphical layout. Early stages of implementation
generated XML based on Strigi/XML format using hardcoded XML snippets and
string formating -- approach often used on webpages with dynamically loaded
content. A string holds a snippet of HTML or XML structure with placeholder.
This placeholder is replaced by either a value or by another already processed
snippet. This way piece by piece the whole scenario is generated. This solution
isn't hard to implement, but brings in poor maintainability and with additional
complexity it looses elegance, readability and can even cause performance
issues. 

Original data of the scenario created by SOWL are stored naturally in
JavaScript object. Using standard JavaScript method {\tt JSON.stringify()} we
can immediately generate JSON serialization of such object. This way we have
structure similar to the original defined by Strigil/XML, but in flexible
structure. Obviously some adaptations are necessary. Nesting is recorded using
the {\tt steps}, the header section is redesigned for the JSON structure. XXX

The original semantics of {\tt onto-elem} and {\tt value-of} was preserved,
only limited to it's basic use. {\tt value-of} serves solely to assign literal
properties. XXX not true anymore ;) 

The final scenario for Use Case 1 XXX looks like this: 

\ref[app:sowljson]



\secc Consequences of conversion to JSON format

XXX We don't have "text content" like XML elements can, but Strigil doesn't really use that

XXX We don't have child nodes.. so we have to keep them in an array, like in {\tt steps} 

XXX But hey, we can use any key to store a substep, not only the array, we
practically use different approach here.. instead of tree structure we have
tree of hash maps. 

The {\tt onto-elem} command benefits exactly from this difference between
XML and JSON.  In original Strigil/XML the {\tt onto-elem} tag allow us to
specify URI of the resulting individual (as commonly denoted by the {\em about}
property), by taking it from from it's ``first child'' which is expected to
be {\tt value-of} tag. Needles to say, this specification lowers robustness as
the position in the XML file isn't enforced by the syntax and can be easily
unintentionally broken by accidental swap of two elements, although it wouldn't
invalidate the files syntax and thus won't be captured by the script parser. In
the JSON format we lack the notion of child elements. Even when we simulate it
as mentioned before, we'd only cause the same indetermination. So instead, we
simply reserve a property named {\tt about} exactly for the described use. 
The same technique is used to specify URL for a {\tt call-template} on it's
{\tt valueof} property. 

XXX TODO might rename these

XXX call-template might need simple steps array property, as it might follow a
list of URLs, not only one

\sec SOWL implementation

XXX 

\sec crOWLer implementation

\secc architecture

XXX include class diagram..?


\secc JavaScript and events support

Special attention have to be payed when dealing with direct interaction with
DOM elements and script execution. WebDriver supports injection and execution
of JavaScript as well as simulation of user interactions like ``click'' on
element or ``back'' and ``forward'' navigation. Even though it brings great
power there are considerations and great limitations to be taken in account. 

WebDriver supports execution of JavaScript directly on webpage loaded in the
driver. This is done by calling {\tt executeScript} or {\tt executeAsyncScript}
function on the {\tt driver} object. First argument of these functions is
string defining content of JavaScript function we want to execute. Header and
actual call of this function will be added for us before it gets attached to
the webpage. We can pass any number of accepted arguments to these functions
and they will be accessible through standard {\tt arguments} object in on the
JavaScript side. Types, corresponding to standard JavaScript types are
supported as arguments: number, boolean, String, WebElement or List of any
combination of the previous~\urlnote{http://goo.gl/Hhwq3l Selenium
JavaScriptExecutor documentation}. The second -- asynchronous version returns
immediately with a {\tt response} object. It provides callback as additional
argument to the JavaScript call. This callback is used for synchronization when
accessing the result on the {\tt response} object from Java.  

XXX TODO better example this one is from
here~\urlnote{http://docs.seleniumhq.org/docs/03_webdriver.jsp\#using-JavaScript}

\begtt
List<WebElement> labels = driver.findElements(By.tagName("label"));
List<WebElement> inputs = (List<WebElement>) ((JavaScriptExecutor)driver).executeScript(
    "var labels = arguments[0], inputs = []; for (var i=0; i < labels.length; i++){" +
    "inputs.push(document.getElementById(labels[i].getAttribute('for'))); } return inputs;", labels);
\endtt

In simple case we can use JavaScript to extend functionality of crOWLer. It
might be used as a complex string formatter, parser for nontrivial values etc. 
In following example it is used to condition on attribute value of an anchor
tag. 

\begtt
WebElement el = driver.findElement(By.cssSelector('a.detail'));
String result = (String) ((JavaScriptExecutor)driver).executeScript(
"var elem = arguments[0];"+
"var href = elem.getAttribute('href');" +
"return (href ==='#' ? window.location.href : href);", el);
\endtt

Previous example is simple, yet if we wanted to cover this use case with our
scenario implementation we would bring a lot of problem-specific XXX balast XXX
into the scenario syntax. We would have to use special syntax for obtaining
current URL and for conditioning on values. Following code demonstrates how this 
functionality might look if it was covered only by scenario syntax without usage
of JavaScript. The {\tt getCurrentUrl} function is inspired by Strigil. 

\begtt
{
  command: "condition", 
  condition: "eq", 
  param: "#", 
  value: {
    commad: "value-of", 
    selector: "a.detail", 
  }
  onfalse: {
    command: "function", 
    value: "getCurrentUrl", 
  }
}
\endtt

We've declared the {\tt condition} command with implementation of {\em eq}
operator (and probably several other un/equality operators) and the {\tt
function} command with implementation of {\em getCurrentUrl} which, again,
probably isn't the last function to be implemented. All this would require
update of the scenario parser, the implementation for commands and all their
attributes and thus update of the whole backend every time, new functionality is
needed. The advantage is, that user doesn't have to know JavaScript and
understand how it is called in WebDriver. It is discutable if XXX

With use of JavaScript it might look as follows: 

\begtt
{
  command: "value-of", 
  selector: "a.detail", 
  exec: "var href = elem.getAttribute('href');" +
        "return (href ==='#' ? window.location.href : href);"
}
\endtt

In this case we embedded only the {\tt value-of} with a single attribute that
takes JavaScript. From there we have technically unlimited power for extending
the functionality of the crOWLer without need of changing the Java
implementation. 

Note that the first line of the original JavaScript was omitted: 

{\tt var elem = arguments[0];} 

It can be automatically prepended every time, we exec JavaScript on a single
DOM element. It is a simple helper and doesn't interfere with anything XXX (as
we can redefine variable as many times as we want). Similarly we could
predefine variable {\tt value} when passing a string or number to a JavaScript
function. 

But with great power comes a great current squared times
resistance~\urlnote{http://www.xkcd.com/643/} XXX. With usage of JavaScript as
suggested in previous paragraphs we have to take in account two major
considerations. 

Firstly, JavaScript function can accept any number of parameters and return an
arbitrary value. In both cases, the parameters and the return value, we can be
of any of the allowed types (as JavaScript isn't strongly typed language). We
thus have to specify what exact parameters are being passed to a function and
what result is expected. We also have to implement a robust way of controlling
and properly define a fallback-on-error behavior. This is especially important
as we might want to use JavaScript function not only as a string filter, but also
for example as a universal selector where we struggle with classical selectors. 
Any additional use have to be described separately before it can be universally
used. 

More importantly, there is the second consideration. Any DOM element is
accessible from the JavaScript function. When this element is modified or even
removed, it becomes invalid from the Java context. The same applies for
operations on the whole page. When a link is followed, the original DOM tree 
is dropped and all references are lost. 

To specify the behavior during this issue, below you can see a simple test.
When link is clicked, the webdriver follows the link in current window and the
reference to the original DOM is lost. 

\begtt
WebDriver wd = new FirefoxDriver();
wd.navigate().to("http://www.inventati.org/kub1x/t/");
WebElement a = wd.findElement(By.cssSelector("a"));
System.out.println(a.getText()); // Prints "detail"
a.click();
System.out.println(a.getText());
// throws org.openqa.selenium.StaleElementReferenceException:
// { "errorMessage":"Element does not exist in cache", ... }
\endtt

In crOWLer, we can now distinguish between two ways of ascending to another HTML page: 

\begitems \style n
  * using {\tt call-template} command
  * using JavaScript or user event such as ``click'' or ``back''
\enditems

The {\tt call-template} is always called on an URL and always creates new web
context, keeping the original one untouched. It actually behaves like call
stack, so when we return from the template call, we can follow on the original
DOM tree. Just to note: compared to corresponding Strigil command, crOWLer
persists the ontological context throughout this call, and so we can relate to it
when assigning properties. 

Direct interaction with current window in any way that changes page location
will, however, irreversibly invalidate all the elements of current DOM. This
doesn't have to mean we can't use this functionality all together. Probably the
best solution would be to only allow DOM modifying operations on the bottom
level of templates (i.e. within the {\tt steps} property of the {\tt template}
command in scenario). At this place we only hold the {\tt body} of current
document and as such we can simply replace it with the newly loaded content. 
In the original crOWLer implementation, this would be the spot between two
``Initial Definitions''. 

Even thought the JavaScript is sandboxed in WebDriver, it is still running in a
browser in your computer and could technically submit some data on a web.
Security issues haven't been considered so far, but might become a point of
interest when we take in account an option of obtaining and executing scenarios
from unknown sources. 


\secc Implemented and unimplemented capabilities

XXX

