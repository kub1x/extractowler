\label[chap:knowledge]
\chap Principles and technologies

In following chapter we will provide basic information about technologies of
Semantic Web and Knowledge Representation. The terminology often used in the
field will be defined to help full understanding before we proceed to the
design and implementation. 


\label[sec:semweb]
\sec Technology of Semantic Web

\midinsert \clabel[pic:semweb-logo]{Logo of Semantic Web}
\picw=7cm \cinspic semweb-logo.png
\caption/f Logo of Semantic Web
\endinsert

Wikipedia defines Semantic Web as a collaborative movement led by international
standards body the World Wide Web Consortium (W3C)~\cite[wiki:semweb]. W3C
itself defines Semantic Web as a technology stack to support a ``Web of data,''
as opposed to ``Web of documents,'' the web we commonly know and
use~\cite[w3c:semweb]. Just like with ``Cloud'' or ``Big Data'' the proper
definition tends to vary, but the notion remains the same. It is collaborative
movement led by W3C and it does define a technology stack. It also includes
users and companies using this technology and the data itself. Technologies
and languages of Semantic Web such as RDF, RDFa, OWL, SPARQL are well
standardized and will be described in following sections of this chapter. 

As a general logical concept of the technology, languages of Semantic Web are
designed to describe data and metadata, give them unique identifiers -- so that
we can address them -- and form them into oriented graphs. The metadata part
define a schema of types (or classes) and properties that both can be assigned
to data and also relations between this types and properties themselves.
Wrapped together this metainformation is being presented in a form of {\em
ontology}.  When some data are annotated by resources from such an ontology we
gain power to {\em reason} on this data, i.e. resolve new relations based on
known ones, and also to {\em query} on our data along with any data annotated
using the same ontology. 

On low level of the implementation we deal with simple {\em oriented graph}.
The graph structure is defined in a form of {\em triples}. Each triple consists
of three parts: {\em subject}, {\em predicate} and {\em object}, which all are
simply {\em resources} listed by their identifiers (URIs). In this very
general form we can express basically any relationship between two resources.
On a level of classes and properties, we can define hierarchies,  or set a
class as a domain of some property. On lower, more concrete level we can assign
a type to an {\em individual}. On a level of ontologies, in a way a
``meta--meta'' level, we can specify for instance an author, description and date
it was released. Each of the relations is described using triples and together
form one complex graph. 


\sec Linked Data

Wikipedia defines Linked Data as ``a term used to describe a recommended best
practice for exposing, sharing, and connecting pieces of data, information, and
knowledge on the Semantic Web using URIs and RDF.'' Just like Semantic Web it is
a phenomena, a community, a set of standards created by this community, tools
and programs implementing these standards and people willing to use these tools
and, of course, the data being presented. Linked data effort strives to solve
the problem of unreachability of majority of the knowledge present on the web,
as it is not accessible in machine readable form, doing so by defining
standards and supporting implementation of those standards. 

To imagine current state of the Linked Data we can take a look on the Linking
Open Data cloud diagram\cite[lod]. The
visualisation~\urlnote{http://lod-cloud.net/versions/2014-08-30/lod-cloud_colored.svg}
contains a node for each ontology and shows known connections between
ontologies. The data originate from \url{http://datahub.io}, a popular web
service for hosting semantic data. Current diagram visualises the state of 
linked data cloud in April 2014. As we can see in the center, many data
resources are linked to DBpedia~\urlnote{http://dbpedia.org}, the semantic data
extracted from Wikipedia. This best describes the notion of Linked data. When
two datasets relate to the same resource, they can be logically linked together
through this connection, as this way they state, they relate to the same thing. 

\midinsert \clabel[pic:lod]{Linking Open Data cloud diagram}
\picw=10cm \cinspic lod.png
\caption/f The Linking Open Data cloud diagram
\endinsert


\sec RDF and RDFS

RDF is a family of specifications for syntax notations and data serialization
formats, meta data modeling, and vocabulary used for it~\cite[wiki:rdf]. 

We will look closely on URI, the resource identifier, vocabularies and
semantics defined by RDF, RDFS, and OWL, and serialization into Turtle and
RDF/XML formats. 

\secc URI

In order to give each resource an unique identifier a Uniform Resource
Identifier is used. This is mostly in a form of URL as we commonly know it as
``web address'' (e.g. {\tt http://www.example.org/some/place\#something}). This
literally specifies address of resource and in many cases can be directly
accessed in order to obtain the related data. In some cases we can use URN as
well. URN as opposed to URL allow us to identify a resource without specifying
its location. This way we can for example use ISBN codes when working with
books and records, or
UUID\urlnote{https://en.wikipedia.org/wiki/Uniform_resource_identifier}, a
Universally Unique Identifier widely used to identify data instances of any
kind. 


\secc RDF and RDFS vocabulary

In order to work with data properly, RDF(S) vocabulary defines several basic
resources along with their semantics.

These are the basic building blocks of our future RDF graphs. The semantics
defined in the specification \fnote{Major part of the vocabulary is described in appendix
\ref[app:rdfsvocab]} allows us to specify class hierarchy, properties with domain
and range as well as use this structure on individuals and literals. This is
the most general standard that lays under every ontology out there. 


\sec OWL

Additionally to RDF and RDFS the OWL -- Web Ontology Language, is a family of
languages for knowledge representation. OWL extends syntax and semantics of RDF, 
brings in notion of subclasses and superclasses, distinction between {\em datatype
properties} and {\em object properties}, defines transitivity, symmetricity and other
logical capabilities of properties. When querying an OWL ontology, it allow us
to use unions or intersections of classes or cardinality of properties. All this
capabilities come in with well defined semantics. Usage of each feature
brought in by OWL semantics extends requirements on resolver being used for
reasoning on our ontology and brings in necessary computational complexity. 


\sec RDFa

RDFa technology defines a concept of embedding content of a web document defined
in HTML with resources from some ontology. Technically we create a invisible
layer of annotations over the data that turns our content into machine readable
record. This is accomplished by embedding the original HTML with custom
attributes. Tools can be used to visualise this data
\urlnote{http://rdfa.info/play/}. 


\label[sec:sparql]
\sec SPARQL

SPARQL is a semantic query language for data stored in RDF format \cite[wiki:sparql].
Using SPARQL syntax we define a pattern of the RDF graph using triples and as a
result we obtain such a nodes that form a subgraph of the original graph
matching the given pattern. So called SPARQL endpoints are the main entry
points through which users can obtain data from openly available
datasets~\urlnote{http://dbpedia.org/sparql Dbpedia SPARQL
endpoint}\urlnote{http://linkedgeodata.org/sparql LinkedGeoData SPARQL
endpoint}. 

Below you can see a simple example of a SPARQL query that returns a list of all
resources from database that have a rdf:type associated to it. 

\begtt
PREFIX rdf:  <http://www.w3.org/1999/02/22-rdf-syntax-ns#>
PREFIX foaf: <http://xmlns.com/foaf/0.1/>
SELECT ?target ?name
WHERE {
  ?target rdf:type foaf:Person;
  OPTIONAL { ?target foaf:name ?name }
}
\endtt


\sec RDF/XML syntax

RDF/XML is one of formats into which we can serialize our RDF
data~\cite[wiki:rdfxml]. It is a regular XML document containing elements and
attributes from the RDF(S) vocabulary. RDF/XML is one of the most common
formats for RDF data serialization. An example from popular FOAF ontology
can be found in appendix \ref[app:rdfxml]. 



\label[sec:turtle]
\sec Turtle syntax

Turtle syntax is another popular syntax for expressing RDF. %~\cite[wiki:turtle].
It allows an RDF graph to be completely written in a compact and natural text
form, with abbreviations for common usage patterns and
datatypes~\cite[w3c:turtle].  Its syntax suits more naturally to RDF data as
it conforms the triple pattern.  Follows an example about author of this work. 

\begtt
@base <http://kub1x.org/> .
@prefix rdf:  <http://www.w3.org/1999/02/22-rdf-syntax-ns#> .
@prefix rdfs: <http://www.w3.org/2000/01/rdf-schema#> .
@prefix foaf: <http://xmlns.com/foaf/0.1/> .

<#me> a foaf:Person;
      foaf:name "Jakub Podlaha".
\endtt

