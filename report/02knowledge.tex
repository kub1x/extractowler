\chap Knowledge base, principles and technologies

In following chapter I'll provide basic information about technologies of
Semantic Web, and Knowledge Representation. The terminology often used in the
field will be defined and used to help full understanding before we proceed to
the design and implementation. 


\sec Technology of Semantic Web

Wikipedia defines Semantic Web as a collaborative movement led by international
standards body the World Wide Web Consortium (W3C)~\cite[wiki:semweb]. W3C
itself defines Semantic Web as a technology stack to support a "Web of data,"
as opposed to "Web of documents," the web we commonly know and
use~\cite[w3c:semweb]. Just like with "Cloud" or "Big Data" the propper
definition tends to vary, but the notion remains the same. It is collaborative
movement led by W3C and it does define a technology stack. It also includes
users and companies using this technology and the data itself. Technologies
and languages of Semantic Web such as RDF, RDFa, OWL, SPARQL are well
standardized and will be described in following sections of this chapter. 

As a general logical concept of the technology, languages of Semantic Web are
designed to describe data and metadata, give them unique identifiers -- so that
we can address them -- and form them into oriented graphs. The metadata part
define a schema of types (or classes) and properties that both can be assigned
to data and also relations between this types and properties themselves.
Wrapped together this metainformation is being presented in a form of {\em
ontology}.  When some data are anotaded by resources from such an ontology we
gain power to {\em reason} on this data, i.e. resolve new relations based on
known ones, and also to {\em query} on our data along with any data annotated
using the same ontology. 

On low level of the implementation we deal with simple {\em oriented graph}.
The graph structure is defined in a form of {\em triples}. Each triple consists
of three parts: {\em subject}, {\em predicate} and {\em object}, wich all are
simply {\em resources} listed by their identifiers (URI's). In this very
general form we can express basically any relationship between two resources.
On a level of classes and properties, we can define hierarchies,  or set a
class as a domain of some property. On lower, more concrete level we can assign
a type to an {\em individual}. On a level of ontologies, in a way a
"meta--meta" level, we can specify for instance an author, description and date
it was released. Each of the relations is described using triples and together
form one complex graph. 


\sec Linked Data

Wikipedia defines Linked Data as "a term used to describe a recommended best
practice for exposing, sharing, and connecting pieces of data, information, and
knowledge on the Semantic Web using URIs and RDF." Just like Semantic Web it's
a phenomena, a community, a set of standards created by this comunity, tools
and programs implementing these standards and people willing to use these tools
and, of course, the data being presented. Linked data effort strives to solve
the problem of unreachability of majority of the knowledge present on the web,
as it is not accessible in machine readable form, doing so by defining
standards and supporting implementation of those standards. 

To imagine current state of the Linked Data we can take a look on the Linking
Open Data cloud diagram~\urlnote{http://lod-cloud.net}. The visualisation
contains a node for each ontology and shows known connections between
ontologies. The data originate from \url{http://datahub.io}, a popular web
service for hosting sematic data. Current diagram visualises the state of 
linked data cloud in April 2014. As we can see in the center, many data
resources are linked to dbpedia~\urlnote{http://dbpedia.org}, the semantic data
extracted from Wikipedia. This best describes the notion of Linked data. When
two datasets relate to the same resource, they can be logically linked together
through this connection, as this way they state, they relate to the same thing. 

XXX lod.png
XXX The Linking Open Data cloud diagram
XXX \url{http://lod-cloud.net/versions/2014-08-30/lod-cloud_colored.svg}

Some additional resources on Linked Data: 

\begitems
 * \url{http://linkeddata.org/guides-and-tutorials}
 * \url{http://linkeddatabook.com/editions/1.0/}
 * \url{http://lov.okfn.org/dataset/lov/}
\enditems


\sec RDF and RDFS

RDF is a family of specifications for syntax notations and data serialization
formats, meta data modeling, and vocabulary used for it~\cite[wiki:rdf]. 

We will look closely on URI, the resource identifier, vocabularies and
semantics defined by RDF, RDFS, and OWL, and serialization into Turtle and
RDF/XML formats. 

\secc URI

In order to give each resource an unique identifier a Uniform Resource
Identifier is used. This is mostly in a form of URL as we commonly know it as
"web address" (e.g. {\tt http://www.example.org/some/place\#something}). This
literarly specify address of resource and in many cases can be directly
accessed in order to obtain the related data. In some cases we can use URN as
well. URN as opposed to URL allow us to identify a resources without specifying
it's location. This way we can for example use ISBN codes when working with
books and records, or
UUID~\urlnote{https://en.wikipedia.org/wiki/Uniform_resource_identifier} a
Universally Uniqe Identifier widely used to identify data instances of any
kind. 


\secc RDF and RDFS vocabulary

In order to work with data properly RDF(S) vocabulary defines several basic
resources along with their semantics.

These are the basic building blocks of our future RDF graphs. The semantics
defined in the specification and slightly described here \ref[rdfsvocab] alow
us to specify class hierarchy, properties with domain and range as well as use
this structure on individuals and literals. This is the most general standard
that lays under every ontology out there. 

\midinsert \clabel[rdfsvocab]{RDF and RDFS vocabulary}
\ctable{rl}{
\hfil resource   & description  \crl \tskip4pt
  rdf:type       & a property used to state that a resource is an instance of a class \cr
                 & a commonly accepted qname for this property is "a" \cr
  rdfs:Resource  & the class of everything; all things described by RDF are resources \cr
  rdfs:Class     & declares a resource as a class for other resources \cr
  rdfs:Literal   & literal values such as strings and integers \cr
                 & property values such as textual strings are examples of RDF literals\cr
                 & literals may be plain or typed \cr
  rdfs:Datatype  & the class of datatypes\cr
                 & rdfs:Datatype is both an instance of and a subclass of rdfs:Clas \cr
                 & each instance of rdfs:Datatype is a subclass of rdfs:Literal \cr
  rdf:XMLLiteral & the class of XML literal values; rdf:XMLLiteral is an instance \cr
                 & of rdfs:Datatype (and thus a subclass of rdfs:Literal) \cr
  rdf:Property   & the class of properties \cr
  rdfs:domain    & (of an rdf:predicate) declares the class of the subject in a triple \cr
                 & whose second component is the predicate \cr
  rdfs:range     & (of an rdf:predicate) declares the class or datatype of the object in a triple \cr
                 & whose second component is the predicate \cr
  rdfs:subClassOf    & allows to declare hierarchies of classes \cr
  rdfs:subPropertyOf & an instance of rdf:Property that is used to state \cr
                     & that all resources related by one property are also related by another \cr
  rdfs:label     & rdf:Property used to provide a human-readable version of a resource's name \cr
  rdfs:comment   & rdf:Property used to provide a human-readable description of a resource \cr
}
\caption/t RDF and RDFS vocabulary
\endinsert


\sec OWL

Additionally to RDF and RDFS the OWL -- Web Ontology Language, is a family of
languages for knowledge representation. OWL extends syntax and semantics of RDF, 
brings in notion of subclasses and superclasses, distinction between datatype
properties and object properties, defines transitivity, symetricity and other
logical capabilities of properties. When querying an OWL ontology, it allow us
to use unions or intercections of classes or cardinality of properties. All this
capabilities comes in with well defined semantics. Usage of each feature
brought in by OWL semantics extends requirements on resolver being used for
reasonging on our ontology and brings in necessary computational complexity. 

Including some more readings on OWL: 

\begitems
 * \url{http://www.w3.org/TR/owl2-primer/}
 * \url{https://en.wikipedia.org/wiki/Web_Ontology_Language}
 * \url{http://www.w3.org/TR/2012/REC-owl2-quick-reference-20121211/}
\enditems


%\sec Ontology repositories
%
%\begitems
% * \url{http://www.w3.org/wiki/Ontology_repositories}
%\enditems


\sec RDFa

RDFa technology defines a concept of embeding content of a web document defined
in HTML with resources from some ontology. Technically we create a invisible
layer of anotatins over the data that turns our content into machine readable
record. This is accomplished by embedding the original HTML with custom
attributes. Tools can be used to visualise this data
\urlnote{http://rdfa.info/play/}. 


\sec SPARQL

Is a semantic query language for data stored in RDF format \cite[wiki:sparql].
Using SPARQL syntax we define a pattern of the RDF graph using triples and as a
result we obtain such a nodes that form a subgraf of the original graph that
match the given pattern. So called SPARQL endpoints are the main entry points
through which users can obtain data from openly available
datasets~\urlnote{http://dbpedia.org/sparql Dbpedia SPARQL
endpoint}\urlnote{http://linkedgeodata.org/sparql LinkedGeoData SPARQL
endpoint}. 

Below you can see a simple example of a SPARQL query that returns list of all
resources from database that have a rdf:Type associated to it. 

\begtt
PREFIX rdf:  <http://www.w3.org/1999/02/22-rdf-syntax-ns#>
SELECT ?target ?type
WHERE {
  ?target rdf:Type ?type;
}
\endtt


\sec RDF/XML syntax

RDF/XML is one of formats into which we can serialize our RDF data~\urlnote{https://en.wikipedia.org/wiki/RDF/XML}. It is a
regular XML document containing elemens and attributes from the RDF(S)
vocabulary. RDF/XML is one of the most common formats for RDF data
serialization. 


\sec Turtle syntax

Turtle syntax is another popular syntax for expressing
RDF~\urlnote{https://en.wikipedia.org/wiki/Turtle_(syntax)}. It's syntax suits 
more naturally to RDF data as it conforms the triple pattern. 


